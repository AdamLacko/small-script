%\def\psfonts{}

\input manual

\vsize=7.1 true in        % page approximately as high as it is wide
\hsize=5.35 true in
\marginwidth=0.93 true in

\ifx \psfonts \undefined
  %\voffset=-0.25 true in % page aligned on top
  \voffset=3.07 true in   % page aligned on bottom
\else
  \voffset=2.25in % for distiller (set page size to 21.0 x 23.0 cm)
  \special{"[/PageMode /UseOutlines /DOCVIEW pdfmark}%
\fi

\pageno = -1

% TITLE PAGE

\vskip30pt
\centerline{\secfont the}
\vskip -3pt
\centerline{\doctitlefont Small}
{\def\ {\hskip 0pt}
 \centerline{\secfont b\ o\ o\ k\ l\ e\ t}
}
\bigskip
\centerline{\doctitlefont Implementor's Guide}
\vskip30pt
\centerline{\subsecfont October 2003}

\vfill  % fill up to the rest of the page
\readtocfile
\vskip10pt \hrule \smallskip \hrule height 1.5pt \smallskip
\rightline{\it ITB CompuPhase}
\eject

% INSIDE PAGE

\headline={\vbox{%
  \tenrm\ifodd\pageno\hfil\folio\else\folio\hfil\fi
  \strut \hrule }}%

\begingroup
\parskip=0pt\sevenrm \everypar{\hangindent=20pt}
``Java'' is a trademark of Sun Microsystems, Inc.

``Microsoft'' and ``Microsoft Windows'' are registered trademarks of
Microsoft Corporation.

``CompuPhase'' is a registered trademarks of ITB CompuPhase.

\endgroup

~\vfill

Copyright \copyright\ 1997--2003, ITB CompuPhase; Brinklaan 74-b, 1404GL Bussum,
The Netherlands (Pays Bas); telephone:~(+31)-(0)35~6939~261 \lbreak
%fax:~(+31)-(0)35 6939 293 \lbreak
e-mail: info@compuphase.com, CompuServe:~100115,2074 \lbreak
WWW: http://www.compuphase.com

The information in this manual and the associated software are provided ``as
is''. There are no guarantees, explicit or implied, that the software and the
manual are accurate.

Requests for corrections and additions to the manual and the software can be
directed to ITB CompuPhase at the above address.

{\font\smallfont=cmr7\smallfont Typeset with \TeX\ in the ``Computer Modern''
 and ``Pandora'' typefaces at a base size of 11 points.}
\eject


% BODY

\pageno = 1
\headline={%
          \ifhardpage
              \let\firstmark=\empty
              \let\botmark=\empty
              \let\topmark=\empty
          \fi
          %
          \vbox{\line{%
          \tenrm
          \ifodd\pageno \hfil{\it\botmark}\quad\llbullet\quad\folio
          \else         \folio\quad\llbullet\quad{\it\firstmark}\hfil
          \fi
          \strut}%
          %
          \ifhardpage
              \global\hardpagefalse
          \else
              \hrule
          \fi
          %
         }}%

\font\dingbats=bbding
\def\dingbatseparator{%
  \vskip 5pt plus 3pt minus 1pt\relax
  {\centerline{{\dingbats\char 113}} \vskip -\parskip}%
  \vskip 5pt plus 3pt minus 1pt\goodbreak
}
\halflineskip


\chapter{Introduction}

``\Small'' is a simple, typeless, 32-bit extension language with a C-like syntax.
The language and features are described in the companion booklet with
the sub-title ``The Language''. This ``Implementor's Guide'' discusses how
to embed the \Small\ scripting language in a host application.

The \Small\ toolkit consists of two major parts:
the compiler takes a script and converts it to P-code (or ``bytecode''),
which is subsequently executed on an abstract machine (or ``virtual machine'').
\Small\ itself is written mostly in the C programming language (there are a
few files in assembler) and it has been ported to Microsoft Windows, Linux,
PlayStation 2 and the XBox. When embedding \Small\ in host applications that are
not written in C or \Cpp, I suggest that you use the AMX DLLs under Microsoft
Windows.

\dingbatseparator

There is a short chapter on the compiler. Most applications execute the compiler
as a standalone utility with the appropriate options. Even when you link the
compiler into the host program, its API is still based on options as if they
were specified on the command line.

The abstract machine is a function libary. The chapter devoted to it contains
several examples for embedding the abstract machine in a host application, in
addition to a reference to all API functions.

Appendices, finally, give compiling instructions for various platforms and
background information ---amongst others the debugger interface and the
instruction set.


\input tools \vfill\eject

\writetocentry{chapter}{Appendices} \pdfbookmark{Appendices}
\input appimp

\forceoddpage   % make sure the index starts at an odd page
\chapter{Index}
\sidx[see]{Bytecode}{P-code}
\sidx[see]{MASM}{Microsoft MASM}
\sidx[see]{NASM}{Netwide Assembler}
\sidx[see]{TASM}{Borland TASM}
\sidx[see]{Virtual Machine}{Abstract \midtilde}
\sidx[see]{WASM}{Watcom WASM}

\bigskip
\beginlist{1em}\compactlist
\list{\lbullet} Names of persons (not products) are in {\it italics}.
\list{\lbullet} Function names, constants and compiler reserved words are
  in |typewriter font|.
\endlist
\bigskip

\begingroup\parskip=0pt plus 1pt
\hfuzz=3pt
\readindexfile{i}
\endgroup

\bye

