%\def\psfonts{}

\input manual

\vsize=7.1 true in        % page approximately as high as it is wide
\hsize=5.35 true in
\marginwidth=0.93 true in

\ifx \psfonts \undefined
  %\voffset=-0.25 true in % page aligned on top
  \voffset=3.07 true in   % page aligned on bottom
\else
  \voffset=2.25in % for distiller (set page size to 21.0 x 23.0 cm)
  \special{"[/PageMode /UseOutlines /DOCVIEW pdfmark}%
\fi

\pageno = -1

% TITLE PAGE

\vskip30pt
\centerline{\secfont the}
\vskip -3pt
\centerline{\doctitlefont Small}
{\def\ {\hskip 0pt}
 \centerline{\secfont b\ o\ o\ k\ l\ e\ t}
}
\bigskip
\centerline{\doctitlefont The Language}
\vskip30pt
\centerline{\subsecfont October 2003}

\vfill  % fill up to the rest of the page
\readtocfile
\vskip10pt \hrule \smallskip \hrule height 1.5pt \smallskip
\rightline{\it ITB CompuPhase}
\eject

% INSIDE PAGE

\headline={\vbox{%
  \tenrm\ifodd\pageno\hfil\folio\else\folio\hfil\fi
  \strut \hrule }}%

\begingroup
\parskip=0pt\sevenrm \everypar{\hangindent=20pt}
``Java'' is a trademark of Sun Microsystems, Inc.

``Microsoft'' and ``Microsoft Windows'' are registered trademarks of
Microsoft Corporation.

``CompuPhase'' is a registered trademarks of ITB CompuPhase.

\endgroup

~\vfill

Copyright \copyright\ 1997--2003, ITB CompuPhase; Brinklaan 74-b, 1404GL Bussum,
The Netherlands (Pays Bas); telephone:~(+31)-(0)35~6939~261 \lbreak
%fax:~(+31)-(0)35 6939 293 \lbreak
e-mail: info@compuphase.com, CompuServe:~100115,2074 \lbreak
WWW: http://www.compuphase.com

The information in this manual and the associated software are provided ``as
is''. There are no guarantees, explicit or implied, that the software and the
manual are accurate.

Requests for corrections and additions to the manual and the software can be
directed to ITB CompuPhase at the above address.

{\font\smallfont=cmr7\smallfont Typeset with \TeX\ in the ``Computer Modern''
 and ``Pandora'' typefaces at a base size of 11 points.}
\eject


% BODY

\pageno = 1
\headline={%
          \ifhardpage
              \let\firstmark=\empty
              \let\botmark=\empty
              \let\topmark=\empty
          \fi
          %
          \vbox{\line{%
          \tenrm
          \ifodd\pageno \hfil{\it\botmark}\quad\llbullet\quad\folio
          \else         \folio\quad\llbullet\quad{\it\firstmark}\hfil
          \fi
          \strut}%
          %
          \ifhardpage
              \global\hardpagefalse
          \else
              \hrule
          \fi
          %
         }}%

\font\dingbats=bbding
\def\dingbatseparator{%
  \vskip 5pt plus 3pt minus 1pt\relax
  {\centerline{{\dingbats\char 113}} \vskip -\parskip}%
  \vskip 5pt plus 3pt minus 1pt\goodbreak
}
\halflineskip


\chapter{Introduction}

``\Small'' is a simple, typeless, 32-bit extension language with a C-like syntax.
Execution speed, stability, simplicity and a small footprint were essential
design criterions for both the language and the interpreter\slash abstract
machine that a \Small\ program runs on.

An application or tool cannot do or be {\it everything\/} for {\it all\/}
users. This not only justifies the diversity of editors, compilers, operating
systems and many other software systems, it also explains the presence of
extensive configuration options and macro or scripting languages in
applications. My own applications have contained a variety of little
languages; most were very simple, some were extensive\dots\ and most needs could
have been solved by a general purpose language with a special purpose library.

The \Small\ language was designed as a flexible language for manipulating objects
in a host application. The tool set (compiler, abstract machine) were written
so that they were easily extensible and would run on different software\slash
hardware architectures.

\dingbatseparator

%\sidx{Small C} \nameidx{Cain, Ron} \nameidx{Hendrix, James} \sidx{Dr.~Dobb's Journal}
%Many years ago, I retyped the ``Small C'' compiler from Dr.~Dobb's Journal, by
%Ron Cain and James Hendrix. Having just grasped the basics of the C language,
%working on the Small C compiler was a learning experience of its own. The
%compiler, as published, generated code for a 8080 assembler. The first
%modification I needed to make was to adapt it to the 8086 processor. Through
%the years that I used it (to write low level system software) I expanded the
%compiler with new features and fixed many details. Eventually, as I was moving
%towards bigger applications in more conventional environments, the Small C
%compiler was replaced by main-stream development environments.
%
%In early 1998, I was looking for a scripting language for an animation
%toolkit. Among the languages that I evaluated were Lua, \smallcaps{BOB},
%Scheme, \smallcaps{REXX}, Java, ScriptEase and Forth. None of these languages
%covered my requirements completely. I have always felt that the C language
%is a flexible and convenient language whose basics can be mastered in a week.
%While experimenting with Quincy (from Al Stevens\nameidx{Stevens, Al}), I
%decided that a simplified C would probably be a good fit. I dusted off Small C.

\sidx{Small C} \nameidx{Cain, Ron} \nameidx{Hendrix, James}
\Small\ is a descendent of the original Small C by Ron Cain and James Hendrix,
which at its turn was a subset of C. Some of the modifications that I did to
Small C, e.g.\ the removal of the type system and the substitution of pointers
by references, were so fundamental that I could hardly call my language a
``subset of C'' or a ``C dialect'' anymore. Therefore, I stripped off the ``C''
from the title and kept the name ``\Small''.

\sidx{Dr.~Dobb's Journal}
I am indebted to Ron Cain and James Hendrix (and more recently, Andy Yuen),
and to Dr.~Dobb's Journal to get this ball rolling. Although I must have
touched nearly every line of the original code multiple times, the Small C
origins are still clearly visible.

\dingbatseparator

A detailed treatise of the design goals and compromises is in appendix
\refcust{Rationale}; here I would like to summarize a few key points.
As written in the previous paragraphs, \Small\ is for customizing applications,
not for writing applications. \Small\ is weak on data structuring because
\Small\ programs are intended to manipulate objects (text, sprites, streams, queries,
\dots) in the host application, but the \Small\ program is, {\it by intent},
denied direct access to any data outside its abstract machine. The only means
that a \Small\ program has to manipulate objects in the host application is by
calling subroutines ---so called ``native functions''--- that the host
application provides.

\Small\ is flexible in that key area: {\it calling functions}. \Small\ supports default
values for any of the arguments of a function (not just the last),
call-by-reference as well as call-by-value, and ``named'' as well as ``positional''
function arguments. \Small\ does not have a ``type checking'' mechanism, by
virtue of being a typeless language, but it {\it does\/} offer in replacement
a ``classification checking'' mechanism, called ``tags''. The tag system is
especially convenient for function arguments because each argument may specify
multiple acceptable tags.

For any language, the power (or weakness) lies not in the individual features,
but in their combination. For \Small, I feel that the combination of default
values for function arguments in combination with named arguments blend together
to a very convenient way to call functions ---and indirectly, a convenient way
to manipulate objects in the host application.

% ??? the ambitions of Small
%     * named parameters in combination with default arguments for everything
%       -> name/value pairs, property settings
%
%       --------------------------------
%
%     * maximum error checking and debugging support

%\dingbatseparator
%
%This booklet tries to unite two books:
%\beginlist{1em}\compactlist
%\list{\lbullet} a manual for the Small compiler and the abstract machine that I
%  wrote;
%\list{\lbullet} a definition of the Small language, independent of the current
%  implementation.
%\endlist
%
%These goals reflect the two main parts of the booklet entitled:
%``Small: the language'' and ``Small: the compiler''. The third part of the
%booklet, ``Appendices'', provides relevant supplementary information and a
%rationale for the design.

\input language \vfill\eject
%\input tools \vfill\eject

\writetocentry{chapter}{Appendices} \pdfbookmark{Appendices}
\input applang

\forceoddpage   % make sure the index starts at an odd page
\chapter{Index}
\sidx[see]{Alias}{External name}
\sidx[see]{Bytecode}{P-code}
\sidx[see]{Virtual Machine}{Abstract \midtilde}

\bigskip
\beginlist{1em}\compactlist
\list{\lbullet} Names of persons (not products) are in {\it italics}.
\list{\lbullet} Function names, constants and compiler reserved words are
  in |typewriter font|.
\endlist
\bigskip

\begingroup\parskip=0pt plus 1pt
\hfuzz=3pt
\readindexfile{i}
\endgroup

\bye

